\section{Contextualização do Problema e Justificativa}

O presente estudo tem como objetivo analisar a existência de vieses de julgamento na população geral e como esses vieses influenciam as escolhas políticas e econômicas. A pesquisa parte da premissa de que as crenças econômicas dos eleitores são frequentemente enviesadas, resultando em decisões políticas potencialmente subótimas para o desenvolvimento econômico e social.

Além disso, é fundamental examinar a interação entre o Estado e a sociedade civil, considerando como essa dinâmica pode influenciar os vieses de julgamento. A relação entre a autoridade estatal e a capacidade de auto-organização da sociedade pode criar um ambiente que exacerba ou mitiga esses vieses. Nesse sentido, o equilíbrio entre a autoridade estatal e a liberdade individual é crucial para a formação das crenças e decisões políticas e econômicas dos cidadãos \cite{acemoglu2019narrow}.

A formação desses vieses também pode ser influenciada pelo legado histórico de regimes políticos anteriores, moldando a memória coletiva e as atitudes contemporâneas em relação ao governo e às políticas públicas \cite{regime_memory}. A compreensão das estruturas sociais e dos valores herdados é essencial para desvendar como essas influências históricas impactam as percepções e julgamentos atuais.

Ademais, as teorias econômicas e a disseminação de conhecimento econômico desempenham um papel significativo na formação das crenças dos eleitores. A complexidade do pensamento econômico e as dificuldades na transmissão acessível dessas ideias ao público geral podem resultar em interpretações enviesadas que afetam diretamente as escolhas eleitorais e políticas. Neste contexto, as ideias de Bryan Caplan sobre o irracionalismo dos eleitores destacam como as preferências sistematicamente enviesadas podem distorcer o processo democrático e as políticas públicas \cite{The_Myth_of_the_Rational_Voter}.

Assim, este estudo não apenas investiga a presença de vieses de julgamento, mas também explora a interseção entre fatores históricos, sociais e econômicos que contribuem para a formação dessas crenças. Ao compreender essas interações complexas, podemos obter insights valiosos sobre como melhorar a educação econômica e promover decisões políticas mais informadas e eficazes, visando o progresso econômico e social.

Além disso, este trabalho contribui para o enriquecimento de uma nova área de estudo denominada "economia política comportamental". Essa área de pesquisa emergente busca integrar princípios da economia, ciência política e psicologia comportamental para entender melhor como os indivíduos tomam decisões econômicas e políticas e como esses processos podem ser influenciados por diversos fatores contextuais e cognitivos.

\section{Fundamentos Teóricos e Evidências Empíricas}

A análise dos fundamentos teóricos e das evidências empíricas na economia política comportamental é crucial para entender como os indivíduos tomam decisões econômicas e políticas. A economia comportamental, ao integrar conceitos da psicologia, desafia a noção tradicional de que os agentes econômicos são perfeitamente racionais e maximizadores de utilidade. Este campo de estudo examina como fatores cognitivos, emocionais e sociais influenciam as escolhas individuais e coletivas, oferecendo uma visão mais realista do comportamento humano.

Pesquisas têm demonstrado que as decisões dos eleitores e formuladores de políticas são frequentemente influenciadas por vieses cognitivos e heurísticas, que podem levar a escolhas subótimas tanto no contexto econômico quanto no político. Esses vieses são moldados por diversos fatores, incluindo experiências passadas, influências culturais e a estrutura institucional na qual os indivíduos estão inseridos \cite{The_Myth_of_the_Rational_Voter, Systematically_Biased_Beliefs_about_Economics, saee1996}. 

A economia política comportamental emergiu como uma área de estudo promissora que busca entender essas dinâmicas complexas. Ao combinar insights da economia comportamental e da ciência política, essa disciplina oferece ferramentas valiosas para analisar como as crenças e atitudes dos eleitores impactam o funcionamento das democracias e a implementação de políticas públicas.


\subsection{O pensamento de economista} 
% o que caracteriza o pensamento de economista segundo a literatura, abordando a racionalidade, a busca por eficiência e a importância da informação.
% como o pensamento do economista se difere das percepções comuns da população comum.

Os economistas no geral operam sob o pressuposto de que os agentes econômicos são racionais e buscam maximizar sua utilidade ou satisfação a partir das escolhas disponíveis, o famoso "Homo Economicus". Esta abordagem implica que os indivíduos tomam decisões de forma lógica e consistente, com base nas informações disponíveis e a partir de uma avaliação cuidadosa das alternativas. É crucial também destacar a importância do pensamento da racionalidade das decisões econômicas, onde se argumenta que o comportamento racional é essencial para a eficiência e a eficácia das políticas econômicas \cite{Hausman_McPherson_Satz_2016}.

A análise de custo-benefício é uma ferramenta central no pensamento dos economistas. Ela permite a avaliação das opções de decisão com base nos custos e benefícios associados, assegurando que as decisões sejam eficientes e justas \cite{KenBinmore2008}. Os economistas utilizam essa abordagem para formular políticas que maximizem o bem-estar social, equilibrando os custos e benefícios de cada intervenção.


A informação desempenha um papel crucial nas decisões econômicas. Os economistas acreditam que decisões de alta qualidade dependem da disponibilidade e do uso eficaz da informação. Eles argumentam que a coleta, a análise e a disseminação de dados são fundamentais para o funcionamento eficiente dos mercados e das políticas públicas \cite{positive_economics_friedman}. No entanto, as limitações cognitivas dos indivíduos e a quantidade limitada de informação disponível podem levar a decisões subótimas, conforme destacado por Daniel Kahneman em sua obra sobre heurísticas e vieses \cite{Judgment_under_Uncertainty}.

O pensamento econômico difere significativamente das percepções comuns da população. Enquanto os economistas enfatizam a racionalidade, a análise de custo-benefício e a importância da informação, a população frequentemente baseia suas decisões em heurísticas e vieses. Essas heurísticas são regras simples e intuitivas que as pessoas usam para tomar decisões rápidas e, embora úteis em muitos contextos, podem levar a erros sistemáticos \cite{Judgment_under_Uncertainty}. A população também é influenciada por crenças infundadas e emoções, o que pode resultar em decisões que não são necessariamente racionais ou eficientes do ponto de vista econômico \cite{thaler2016misbehaving}.

Em resumo, a abordagem de Gary Becker define o que hoje temos por regra atualmente:

\begin{citacao}
    \textit{Acho difil acreditar que a maioria dos eleitores seja sistematicamente enganada quanto aos efeitos de políticas como a de quotas e tarifas de importações que persistem há tempos. Prefiro supor que o eleitor tem expectativas não enviesadas, ao menos, quanto a essas políticas persistentes. Eles talvez superestimem o peso-morto de algumas medidas e subestimem o de outras, mas, em média, eles tem uma ideia correta.
    } \newline \cite{becker1976}
\end{citacao}

\subsection{Economistas e os Viéses de Julgamento nos Eleitores}
% Fazer uma introdução igual Caplan faz no livro dele, falando sobre a importância da democracia e como ela depende de eleitores bem informados e racionais.

A democracia se fundamenta na ideia de que eleitores informados e racionais são capazes de tomar decisões que promovem o bem-estar coletivo e o desenvolvimento sustentável. Em "The Myth of the Rational Voter", Caplan argumenta que a eficácia da democracia depende criticamente da capacidade dos eleitores de avaliar políticas e candidatos de maneira objetiva e informada. Contudo, na prática, muitos eleitores sofrem de vieses de julgamento que distorcem suas percepções e decisões, resultando em escolhas que podem ser prejudiciais para a sociedade como um todo \cite{The_Myth_of_the_Rational_Voter}.

Acemoglu e Robinson, em "The Narrow Corridor", enfatizam que o engajamento político ativo e informado da população é crucial para manter o equilíbrio entre a autoridade estatal e a liberdade individual. Eles argumentam que a capacidade da sociedade de se organizar e participar ativamente nos processos políticos é fundamental para evitar a tirania e promover políticas que refletem verdadeiramente as necessidades e interesses da população \cite{acemoglu2019narrow}.

No entanto, os vieses de julgamento nos eleitores podem comprometer esse ideal democrático. Esses vieses são frequentemente resultado de heurísticas cognitivas, que são atalhos mentais usados para simplificar a tomada de decisão, mas que podem levar a erros sistemáticos. Entre os principais vieses que afetam os eleitores estão:

\begin{enumerate}
    \item Viés antimercado
    \item Viés antiestrangeiro
    \item Viés antitrabalho
    \item Viés pessimista
\end{enumerate}

Atualmente, o estudo da "Economia Política Comportamental" e seus vieses está sendo revitalizado. No entanto, é crucial lembrar que a história do pensamento econômico sempre foi marcada por discussões sobre esses temas e seus correlatos.

Muitos dos famosos economistas do passado, como Adam Smith e Fréderic Bastiat, eram obcecados pela moralidade e pelas crenças teimosas do povo quanto a economia, a sua insistênte resistência  aos principios básicos do custo de oportunidade e a vantagem comparativa \cite{hart2019bastiat,Wells2013,The_Myth_of_the_Rational_Voter}.

Parece que as os economistas tem se esquecido de que a economia é uma ciência social e que as pessoas são seres humanos, com crenças e valores que muitas vezes não se alinham com a lógica econômica. Questões que hoje são levantadas como novas, já foram a tempos levantadas por economistas como Bastiat, que em sua obra "O que se vê e o que não se vê" já discutia sobre a dificuldade de se perceber os custos ocultos das políticas públicas e como elas se comportam na subjetividade \cite{hart2019bastiat}.

Estudar mais a fundo esses viéses presentes nos eleitores, tendo como um grande aliado a história do pensamento economico se faz mais necessário que nunca. L embrando que o problema do tema não é que os economistas não tem nada a dizer sobre o assunto, mas sim que eles tem muito a dizer porém relutam a ir em público e arriscar a sua credibilidade científica. Se fosse possível superar essa relutância teríamos muito a dizer \cite{The_Myth_of_the_Rational_Voter}. Gustavo Franco pode elucidar essa parte como ninguém:

\begin{citacao}
    \textit{[...] Tenha claro, por favor, que não há problema nenhum em atacar a sabedoria estabelecida [...]. Mas não perca de vista que ir contra o senso comum é um esporte radical: há muito risco e, se você errar, vai acabar no hospital. Sempre é preciso provar o que você diz, e costuma ser difícil. [...] Muito cuidado ao atacar a sabedoria estabelecida, pois na maioria das vezes você estará errado. Lembre-se de que o conhecimento que você herda se estabeleceu do trabalho diligente de muitos como você e eu, depois de anos e anos de tentativa, erro e decantação. \newline
    }  \cite{franco2022cartas}
\end{citacao}

Então temos em nossas mãos um cenário muito otimista apenas de revelar o que os economistas já sabem. Poucos economistas contemporâneos se preocupam com a história do pensamento econômico, e isso é um erro, deixando muitas discussões importantes ignoradas ou esquecidas \cite{mark_history}. A história do pensamento econômico é uma disciplina que nos ajuda a entender como as ideias econômicas evoluíram ao longo do tempo e como elas moldaram a sociedade em que vivemos. Ela nos permite ver como as ideias econômicas foram influenciadas por eventos históricos, mudanças políticas e avanços tecnológicos, e como essas ideias continuam a influenciar o pensamento econômico contemporâneo.

Porém o foco aqui são os viéses que não podem ser ignorados, explicando eles com base em no que os economistas acreditam ser o certo ao longo da história documentada e em contraste o que a população no geral acredita. Provas formais dos viéses estarão mais a frente.


\subsubsection{Viés antimercado}
% Descreva como o viés antimercado leva os eleitores a desconfiarem das soluções de mercado, favorecendo intervenções estatais.

O viés antimercado pode ser resumido na tendência de subestimar os benefícios do mercado, de seus mecanismos de mercado e superestimar os custos associados a ele \cite{sowell2000basic,sowell2004applied,The_Myth_of_the_Rational_Voter}. 

O cidadão comum tende a acreditar que o mercado é ineficiente e injusto, tende a ter séria dúvidas de até onde pode confiar e contar com empresas lucrativas para gerar produtos socialmente benéficos. Se foca somente na motivação do lucro da empresa e é deixado de lado a parte da disciplina imposta pelo mercado, que faz com que empresas que não atendem as necessidades do consumidor sejam eliminadas do mercado. Os economistas no geral admitem que, a busca incessante pelo lucro aliada as falhas e imperfeições de mercado podem gerar resultados ruins, não-economistas veem a ganância bem sucedida como algo socialmente prejudicial por si só \cite{The_Myth_of_the_Rational_Voter}.

Shumpeter, em sua obra "Capitalismo, Socialismo e Democracia", define com perfeição esse viés:

\begin{citacao}
    \textit{O capitalismo é julgado por juízes que têm a sentença de morte preparada. Eles darão este veredito, não importa o que a defesa diga; a única coisa que a defesa pode fazer é provocar uma mudança na acusação. \newline
    }  \cite{schumpeter1976capitalism}
\end{citacao}

Essa visão de que o lucro é simplesmente uma forma de transferência, uma exploração e que o mercado é um jogo de soma zero, onde o ganho de um é a perda de outro, é um dos principais fatores que levam os eleitores a desconfiarem das soluções de mercado e a favorecerem intervenções estatais. Vale ressaltar que "transferência" no dialéto econômico é um termo que se refere a um movimento descompromissado de riqueza de um agente para outro, sem que haja um aumento na riqueza total da sociedade. Levando isso como base, podemos concluír que as pessoas tendem a ver os lucros como um presente para os mais ricos. Portanto, a não ser que o governo intervenha limitando os lucros como questão de bom senso, a riqueza não será redistribuída de forma justa e eficiente \cite{The_Myth_of_the_Rational_Voter}. Essa forma de pensar é exatamente o que Thomas Sowell chama de "raciocínio de estágio único", ou seja, considerar apenas as consequências imediatas e óbvias de uma medida, ignorando as consequências indiretas e menos óbvias \cite{sowell2004applied}.

É importante lembrar que o lucro não é uma esmola, mas sim um \textit{quid pro quo}: o lucro é a recompensa por atender as necessidades e desejos dos consumidores de forma eficiente e inovadora. O lucro é o sinal de que uma empresa está gerando valor para a sociedade, alocando os recursos de forma eficiente, e não um mecanismo de exploração. Essa é a lição básica que Adam Smith nos ensinou em "A Riqueza das Nações": a "mão invisível" do mercado, guiada pela busca do lucro, silenciosamente convence os empresários egoístas a servir o bem comum \cite{smith1776inquiry}.

Desde o início da história registrada também temos o lucro aparecendo de forma prejudicial, como o exemplo do lucro sobre o empréstimo de dinheiro, o juros, que era considerado pecaminoso e injusto. Desde a mesopotâmia até a igreja católica medieval, por exemplo, proibindo a cobrança de juros, considerando-a uma forma de exploração e usura \cite{tomasdeaquino_summa_78}. Eugen von Böhm-Bawerk consegue elucidar essa questão de forma clara em seu clássico "Capital e Juro":

\begin{citacao}
    \textit{O credor geralmente é rico e o devedor, pobre, e o primeiro parece um homem odioso que suga o pouco que o pobre tem na forma de juros e que pode aumentar ainda mais sua riqueza supérflua. Não é de se surpreender, contudo, que tanto a Antiguidade quanto a Idade Média Cristã viam com maus olhos os juros. 
    } \newline \cite{von2022capital}
\end{citacao}

Para encerrar esta parte mostrando a visão dos economistas, uma pessoa que ouça ou veja economistas discutindo questões como essa, como Krugman ou Stiglitz, pode ter a impressão de que a parte benéfica do mercado ainda é controversa ou não está acertada \cite{krugman2003great,stiglitz2003roaring,The_Myth_of_the_Rational_Voter}. Nesse ponto, é importante entender que os economistas não estão debatendo se os preços dão incentivos ou se há uma enorme conspiração para manter os preços altos. Eles estão simplesmente explicando como o mercado funciona. Quase todos os economistas reconhecem que o mercado é a melhor maneira de alocar recursos escassos e que a concorrência é a melhor maneira de garantir que os preços sejam justos e que os consumidores sejam protegidos; eles discordam apenas sobre o grau disso.

\subsubsection{Viés antiestrangeiro}
% Descreva como o viés antiestrangeiro leva os eleitores a desconfiarem de acordos comerciais e imigração, favorecendo políticas protecionistas e anti-imigração.

É sempre interessante começar a explicação de um tema começando com uma pergunta fundamental a ser respondida: Estrangeiros? Será que pe mesmo mutuamente benéfico comercializar com eles?

Uma frase de Alan Blinder pode começar a explicar o motivo dessa desconfiança:

\begin{citacao}
    \textit{
        Quando os empregos são escassos, o instinto de autopreservação ganha força e a tentação de culpar a concorrência estrangeira é irresistível. Não só nos Estados Unidos que a mentalidade protecionista ganhou peso. O fato de muitos economistas considerarem míope e um autoboicote o esforço de, por meio do protecionismo, salvar empregos não cabe aqui. Os legisladores estão aí para ganhar votos, não elogios de intelectuais.
     } \newline 
    \cite{blinder1987hard}
\end{citacao}

Para reforçar o discurso, nada melhor que o pai da economia moderna para também citar o que naquela época já era considerado reprovável:

\begin{citacao}
    \textit{
         De que vale a prudência na conduta de todas as famílias se a escassez pode ser enganada num grande reino? Se um país estrangeiro puder nos fornecer uma mercadoria por um preço mais baixo do que o nosso, melhor comprá-la com uma parte do produto da nossa indústria, empregando de forma a termos alguma vantagem.
    } \newline
    \cite{smith1776inquiry}
\end{citacao}

Os economistas criticam o viés antiestrangeiro porque ela não só está errada como também essa visão entra de conflito com a economia mais elementar ensinada nos primeiros meses de graduação. Os livros mais usados, como o de Mankiw, ensinam que o comércio é mutuamente benéfico, que a especialização e a troca aumentam a produtividade e o bem-estar de todos os envolvidos. Nas palavras dele, de seus dez princípios de economia, o número cinco é:

\begin{citacao}
    \textit{
        [...] É fácil se enganar, porém, ao pensar na competição entre países. O comércio entre os Estados Unidos e a China não é como uma competição esportiva, em que um lado ganha e o outro perde. De fato, o oposto é verdadeiro: o comércio entre dois países pode ser bom para ambas as partes.
    } \newline
    \cite{mankiw2020introducao}
\end{citacao}

A Lei das Vantagens Comparativas é o fenômeno que é descrito acima. Essa teoria, desenvolvida por David Ricardo, reforça ainda mais a importância do comércio internacional. Segundo Ricardo, mesmo que um país seja menos eficiente na produção de todos os bens em comparação a outro, ainda é vantajoso para ambos os países se especializarem na produção dos bens nos quais têm uma vantagem comparativa (ou seja, onde têm uma menor desvantagem relativa) e comercializarem entre si. Isso ocorre porque a especialização baseada nas vantagens comparativas maximiza a produção global e o bem-estar econômico \cite{ricardo1817principles}.

Além disso, políticas protecionistas frequentemente resultam em ineficiências econômicas e redução do bem-estar. O livre comércio permite que os países aproveitem suas vantagens comparativas, promovendo a eficiência e o crescimento econômico global \cite{bhagwati2003free}.

Outro exemplo de viés antiestrangeiro é a desconfiança em relação à imigração. Muitos eleitores acreditam que a imigração prejudica a economia local, aumenta a concorrência por empregos e recursos escassos e ameaça a identidade cultural. Os economistas de hoje reconhecem fácilmente os benficios da imigração. A troca de trabalho é praticamente a mesma que a troca de mercadorias. Especíalização e a troca aumentam a produção e o bem-estar de todos os envolvidos. A imigração aumenta a força de trabalho, a produtividade e a inovação, impulsionando o crescimento econômico e a prosperidade.

Em resumo, o viés antiestrangeiro leva os eleitores a desconfiarem de acordos comerciais e imigração, favorecendo políticas protecionistas e anti-imigração. No entanto, os economistas argumentam que o comércio internacional e a imigração são benéficos para a economia, promovendo a eficiência, o crescimento econômico e o bem-estar global.

Para finalizar com uma citação bem humorada, nada melhor que a de Steven Landsburg explicando que o comércio internacional é como uma tecnologia:

\begin{citacao}
    \textit{
        Há duas tecnologias para a produção de carros nos Estados Unidos. Uma delas é a manofatura em Detroit e a outra é a agricultura em Iowa. Todos conhecem a primeira; deixe-me falar da segunda. Primeira você planta as sementes, que são a matéria-prima dos carros. Você espera alguns meses até o trigo crescer. Daí você colhe o trigo, põe em navios e coloca os navios para cruzarem o Oceano Pacífico. Depois de alguns meses, os navios aparecem com Toyotas dentro deles.
    } \newline
    \cite{landsburg2012armchair}
\end{citacao}


\subsubsection{Viés antitrabalho}
% Discuta como o viés anticonservação do trabalho leva a uma resistência às mudanças tecnológicas e inovações. 





\subsubsection{Viés pessimista}
% Descreva como o viés pessimista faz com que os eleitores tenham uma visão negativa sobre a economia, mesmo em tempos de crescimento.


\subsubsection{A influência dos vieses e a armadilha das ideias}
% Falar de como esses vieses se enquadram em um problema econômico? Falar da dinâmica proposta pela noção de que indivíduos podem ter preferências por crenças.


\subsection{Influência da Memória de Regime}
% Discuta como as memórias de regimes passados (democracias, ditaduras, etc.) influenciam as preferências dos eleitores.

\section{Abordagens Metodológicas}
A metodologia da pesquisa será dividida em duas partes principais: a análise dos trabalhos que abordam os viéses de julgamento dos eleitores e a investigação sobre a influência da memória de regime.

\subsection{Análise dos Trabalhos sobre Viéses de Julgamento}

\subsubsection{Estados Unidos da América}

\subsubsection{Portugal}

\subsubsection{Brasil}

\subsection{Investigação sobre a Influência da Memória de Regime}


\section{Interpretação dos Resultados e Implicações}

\section{Conclusão}

